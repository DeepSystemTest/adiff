
\documentclass{article}
\usepackage{algorithm}
\usepackage{amsmath}
\usepackage{algpseudocode}
\usepackage[utf8]{inputenc}

% New definitions
\algnewcommand\algorithmicswitch{\textbf{switch}}
\algnewcommand\algorithmiccase{\textbf{case}}
\algnewcommand\algorithmicassert{\texttt{assert}}
\algnewcommand\Assert[1]{\State \algorithmicassert(#1)}%
% New "environments"
\algdef{SE}[SWITCH]{Switch}{EndSwitch}[1]{\algorithmicswitch\ #1\ \algorithmicdo}{\algorithmicend\ \algorithmicswitch}%
\algdef{SE}[CASE]{Case}{EndCase}[1]{\algorithmiccase\ #1}{\algorithmicend\ \algorithmiccase}%
\algtext*{EndSwitch}%
\algtext*{EndCase}%


\begin{document}


\section{Algorithms}

\subsection{weighted positions}
Three of adiff's strategies (dfs, bfs and uniform) are instances of the following
algorithm that is parameterized by a weight function.

\begin{algorithm}
  \caption{Weighted positions}
  \begin{algorithmic}
  \Procedure{weightedPositions}{weight function $w$, translation unit $tu$, budget $b$}
  \State $\textit{reads} \gets \Call{findAllReads}{tu}$
  \State $\textit{constants} \gets \Call{findAllConstants}{tu}$
  \State $\textit{weights} \gets \text{map $w$ $reads$} $
  \For {$b$ times} 
      \State $r \gets \text{randomly draw from $reads$ with weights $weights$}$
      \State $constant \gets ite(\text{randomBool()}
      , \text{constants.lookup(r.type)}
      , \Call{randomValue}{r.type})$
      \State let $assertion = \texttt{r.variable != constant}$ 
      \State $tu' \gets$ insert(tu, $assertion$, r.position)
      \State $result \gets \text{verify}(tu')$
  \EndFor
  \EndProcedure
\end{algorithmic}
\end{algorithm}
\begin{itemize}
\item `dfs' is `weightedPositions' parametrized by the weight function
  \[
    w_{\text{dfs}}(r) = depth (r.position)
  \]
\item `bfs' is `weightedPositions' parametrized by
  \[
    w_{\text{bfs}}(r) =  1 / depth(r.position)
  \]
\item `uniform-random' is `weightedPositions' parametrized by
  \[
    w_{\text{uniform}}(r) = 1
  \]
\end{itemize}

\newpage
\subsection{Zipper-based algorithms}
In short: A zipper is basically a two-dimensional iterator that can also be used
to modify an object. Whereas a regular iterator only has methods to ask for the
current element and to move the iterator to the next element, our zipper has the
following directions: TODO.
Two of our strategies are based on those zippers.

\begin{algorithm}
\begin{algorithmic}
  \caption{Random Walk}
  \Procedure{RandomWalk}{translation unit $tu$, budget $b$}
  \State $\textit{constants} \gets \Call{findAllConstants}{tu}$
  \State $zipper \gets \Call{makeZipper}{tu}$
  \While {$ b > 0 $}
  \State \Call{MoveRandomDirection}{$zipper$}
  \State $reads \gets \Call{ReadsAt}{zipper}$
  \If {$reads.length == 0$}
  \State \textbf{Continue}
  \EndIf
  \State $r \gets \text{randomly choose from $reads$}$
  \State $constant <- ite(randomBool(), lookupConstants, randomConstant)$
  \State let $assertion = $ \texttt{r.variable != constant}
  \State $tu' \gets \text{zipper.insert($assertion$)}$
  \State $verify(tu')$
  \State $b \gets b - 1$
  \EndWhile
  \EndProcedure
\end{algorithmic}
\end{algorithm}

%%%%%%%%%%%%%%%%%%%%%%%%%%%%%%%%%%%%%%%%%%%%%%%%%%%%%%%%%%%%%%%%%%%%%%%%%%%%%%%%


\begin{algorithm}[ht]
  \caption{'Smart' algorithm}
  \begin{algorithmic}
  \State var $zipper$
  \State var $\textit{constants}$

  \Procedure{Smart}{budget $b$}
  \While {zipper.inCompound()}
    \State zipper.go(Down);
  \EndWhile
  \State $ children \gets \Call{ExploreLevel}{\ }$
  \For {$c \in children$}
      \State $b' \gets \text{a fraction of b, proportional to $c.rating$} $
      \State $\Call{ExplorePosition}{c.position, c.rating, b'}$
  \EndFor
  \EndProcedure
  \vspace{1em}
  \Procedure{ExplorePosition}{position p, rating r, budget b}
  \State {zipper.gotoPosition(p)}
  \If {zipper.currentPosition.isFunctionCall()}
    \State $\Call{ExploreStatement}{b / 4}$
    \State {zipper.followFunctionCall()}
    \State $\Call{Smart}{b / 4}$ \Comment{recurse}
  \Else
    \State $\Call{ExploreStatement}{b / 4}$
  \EndIf
  \If {zipper.go(Down)}
    \State $\Call{Smart}{b / 2}$ \Comment{recurse}
  \EndIf
  \EndProcedure
  \vspace{1em}
  \Procedure{ExploreStatement}{budget b}
  \State {reads $\gets$ zipper.readsAtCurrentPosition()}
  \For {$r \in reads$}
  \If{$b \leq 0$} \State{\textbf{Break}}\EndIf
  
  \State {\textbf{let} asrt = a  assertion with constant from constant pool, or if pool
    is empty a random constant}
  \State {tu' $\gets$ zipper.insertBefore(art)}
  \State {verify(tu')}
  \State {$b \gets b - 1$}
  \EndFor
  \EndProcedure
\end{algorithmic}
\end{algorithm}

Of special interest here is the function $\Call{ExploreLevel}$, as this one
control how much of the budget is alloted  to different parts of the program.

\begin{algorithm}[ht]
  \begin{algorithmic}
  \Procedure{ExploreLevel}{}
   \State {$scores = \emptyset$}
   \State {$usedBudget = 0$}
   
   \Loop
   \State {tu' $\gets$ zipper.insert(assertFalseStmt)}
   \State {(results, conclusion) $\gets$ verify(tu')}
   \State {usedBudget $\gets$ usedBudget + 1 }
   \State {updateMovingAverages(results)}
   \State {d $\gets$ disagreement(conclusion)}
   \State {t $\gets$ irregularity(results.times)}
   \State {scores.add(d + t)}
   
   \If{not(zipper.go(Down))} \State{\textbf{Break;}} \EndIf
   \EndLoop
   \State {\Return (scores, usedBudget)}
  \EndProcedure

   \vspace{1em}
   \Procedure{Disagreement}{conclusion c}
    \Switch{c}
    \Case{StrongAgreement}
    \Return 0.1
    \EndCase
    \Case{WeakAgreement}
    \Return 1.0
    \EndCase
    \Case{Unsoundness} \Return 100
    \EndCase
    \Case{Incompleteness}
    \Return 10
    \EndCase
    \Case{Disagreement}
    \Return 3
    \EndCase
    \EndSwitch
   \EndProcedure

   \vspace{1em}
   \Procedure{irregularity}{timings $ts$}
   \State {proportions = fractions of timing t with its average time } 
   \State {$s \gets \Sigma \{ relativeError(t_1,t_2) \mid t_1,t_2 \in proportions \}$}
   \State {\Return {$s / n^2$}}
   \EndProcedure
\end{algorithmic}
\end{algorithm}

Relative Error:
\[
  relativeError(a,b) &=
  \begin{cases}
    0 & \text{if $a = b = 0$} \\
    \dfrac{| a - b |}{\max (|a|, |b|)}

  \end{cases}

\]

\end{document}